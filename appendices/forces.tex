\chapter{Calculation of Forces}
\label{app:forces}

The calculation of forces from a given potential model is central to the geometry optimisation routines (see section \ref{s:geomopt}) employed in many algorithms in this thesis.
This appendix derives the equations used to calculate the forces for the potentials covered throughout this work.
The notation used will aim to be consistent, and is outlined as follows.
An position vector is denoted by $\vr=\left[r_1\, r_2\right]^{T}$, with:
\begin{align}
	\vrij=\vrj-\vrj\,, \\
	r_{ij}=\abs{\vrij}\,, \\
	\vhrij=\frac{\vrij}{r_{ij}}\,.
\end{align}
The derivative of a function with respect to $\vr$ is then given by $\frac{\partial f\left(\vr\right) }{\partial \vr}=\left[\frac{\partial f\left(\vr\right)}{\partial r_1}\, \frac{\partial f\left(\vr\right)}{\partial r_2}\right]^{T}$.
It therefore follows that:
\begin{align}
	\frac{\partial r_{ij}}{\partial\vrij}=\vhrij\,,\quad \frac{\partial r_{ij}}{\partial \vri}=-\vhrij\,,\quad \frac{\partial r_{ij}}{\partial \vrj}=\vhrij \,.
\end{align}

Angles are denoted by $\theta_{ijk}$, representing the angle between $\vrij$ and $\vrik$. 
It will also be useful to determine the derivative of the cosine of angles with respect to a position vector:
\begin{align}
	\frac{\partial \cos\theta_{ijk}}{\partial \vrj} &= \frac{\partial}{\partial \vrj}\left(\frac{\vrij\cdot\vrik}{r_{ij}r_{ik}}\right) \nonumber \\[0.5em]
	&= \frac{r_{ij}r_{ik}\vrik-\vrij\cdot\vrik\vhrij\vrik}{r_{ij}^2r_{ik}^2} \nonumber \\[0.5em]
	&= \frac{1}{r_{ij}}\left(\vhrik-\vhrij\cdot\vhrik\vhrij\right) \nonumber \\[0.5em]
	&= \frac{1}{r_{ij}}\left(\vhrik-\cos\theta_{ijk}\vhrij\right)\,,
\end{align}
and similarly
\begin{equation}
	\frac{\partial \cos\theta_{ijk}}{\partial \vrk} = \frac{1}{r_{ik}}\left(\vhrij-\cos\theta_{ijk}\vhrik\right)\,.
\end{equation}
These relationships form the basis to derive the forces for the various stretching and angular potentials used in this thesis.

The force on a given particle at $\vri$ is given by the negative derivative of the potential:
\begin{equation}
	\vfi=-\frac{\partial \mathcal{U}}{\partial \vri}\,.
\end{equation}
As forces are conservative, the sum of the forces on all particles must be zero \ie{} for stretching and angular terms respectively:
\begin{align}
	\vfi &= -\vfj\,, \label{eq:fistretch}\\
	\vfi &= -\vfj -\vfk. \label{eq:fiangle}\
\end{align}
In the following sections $\fk$ denotes a force constant and a subscript zero an equilibrium value.

\section{Harmonic Stretching Potential}

The harmonic stretching potential is a simple bonding potential that approximates many atomic potentials at small displacements.
The interaction between two particles at separation $\vrij$ is given by:
\begin{equation}
	\mathcal{U} = \frac{\fk}{2}\left(r_{ij}-r_0\right)^2\,.
\end{equation}
The forces are therefore:
\begin{align}
	\vfj&=-\frac{\partial \mathcal{U}}{\partial r_{ij}} \frac{\partial r_{ij}}{\partial \vrj} = -\fk\left(r_{ij}-r_0\right)\vhrij\,,
\end{align}
with $\vfi$ given by equation \eqref{eq:fistretch}.

\section{Quartic Stretching Potential}

The quartic stretching potential is related to the harmonic potential, but is even more computationally efficient as there is no square root operations are required.
The interaction between two particles at separation $\vrij$ is given by:
\begin{equation}
	\mathcal{U} = \frac{\fk}{4}\left(r_{ij}^2-r_0^2\right)^2\,.
\end{equation}
The forces are therefore:
\begin{align}
	\vfj&=-\frac{\partial \mathcal{U}}{\partial r_{ij}} \frac{\partial r_{ij}}{\partial \vrj} = -\fk\left(r_{ij}^2-r_0^2\right)\vrij\,,
\end{align}
with $\vfi$ given by equation \eqref{eq:fistretch}.

\section{Harmonic Cosine Angle Potential}

In analogue with the stretching potential, the harmonic cosine angle is a elegant yet simple form angular potential, utilising the cosine function to reduce overheads when calculating angles.
The interaction between three particles with angle $\theta_{ijk}$ is given by:
\begin{equation}
	\mathcal{U} = \frac{\fk}{2}\left(\cos\theta_{ijk}-\cos\theta_0\right)^2\,.
\end{equation}
The forces are therefore:
\begin{align}
	\vfj&=-\frac{\partial \mathcal{U}}{\partial \cos\theta_{ijk}} \frac{\partial \cos\theta_{ijk}}{\partial \vrj} \nonumber \\ 
	&=-\frac{\fk}{r_{ij}}\left(\cos\theta_{ijk}-\cos\theta_0\right)\left(\vhrik-\cos\theta_{ijk}\vhrij\right) \,,
\end{align}
with $\vfk$ having an analogous form and $\vfi$ given by equation \eqref{eq:fiangle}.

\section{Restricted Bending Potential}

The restricted bending (ReB) potential is a modification on the harmonic cosine angle potential which diverges at $\theta_{ijk}=0,\pi$, ensuing angles cannot become reflex.
The interaction between three particles with angle $\theta_{ijk}$ is given by:
\begin{equation}
	\mathcal{U} = \frac{\fk}{2}\frac{\left(\cos\theta_{ijk}-\cos\theta_0\right)^2}{\sin^2\theta_{ijk}}\,.
\end{equation}
The forces are therefore:
\begin{align}
	\vfj&=-\frac{\partial \mathcal{U}}{\partial \cos\theta_{ijk}} \frac{\partial \cos\theta_{ijk}}{\partial \vrj} \nonumber \\ 
	&=-\frac{\fk}{r_{ij}\sin^4\theta_{ijk}}\left(\cos\theta_{ijk}-\cos\theta_0\right)\left(1-\cos\theta_{ijk}\cos\theta_0\right)\left(\vhrik-\cos\theta_{ijk}\vhrij\right) \,,
\end{align}
with $\vfk$ having an analogous form and $\vfi$ given by equation \eqref{eq:fiangle}.

\section{Keating Potential}

The Keating potential combines the quartic stretching potential with a computationally efficient angle potential of the form:
\begin{equation}
	\mathcal{U} = \frac{\fk}{2}\left(\vrij\cdot\vrik-r_0^2\cos\theta_0\right)^2\,
\end{equation}
for three particles with an angle given by $\vrij$ and $\vrik$.
The forces are therefore:
\begin{align}
	\vfj&=-\frac{\partial \mathcal{U}}{\partial \vrj} \nonumber \\ 
	&=-\fk\left(\vrij\cdot\vrik-r_0^2\cos\theta_0\right)\vrik \,,
\end{align}
with $\vfk$ having an analogous form and $\vfi$ given by equation \eqref{eq:fiangle}.

