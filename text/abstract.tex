Investigations into \td{} network\--forming materials have renewed importance due to the experimental synthesis of ultra\--thin films of carbon, silica and aluminosilicate glasses.
Microscopy on the disordered phases of these materials reveals a percolating ring structure, identifying them as candidates for range of technologically useful materials, with applications including catalysis and gas separation.
The ability to characterise, understand and control the structure of the pore landscape in such materials is therefore central to their further development.

The overall aim of this thesis is to improve the simulation and characterisation of \td{} disordered materials.
The key is to accurately quantify and reproduce the ring structure, described by the ring size distribution and ring\--ring correlations.
This will be achieved by relating previous empirical laws (namely \lm's law and the \aw{} law), to well\--defined metrics in network science, and developing a variety of \mc{} methods to generate networks with controllable ring structure and topologies.

The first of these methods will be a sequential growth algorithm to construct triangle rafts (a proxy for silica bilayers), which will be shown to produce configurations commensurate with those from experiment.
Following from this, a modified bond switching algorithm will be used to investigate more generic systems, spanning a variety of atomic coordination environments, potential models and topologies. 
Fundamental relationships between these measures and the ring structure will be demonstrated by systematically varying the system parameters.
In addition, hard particle \mc{} techniques will be used to generate Voronoi diagrams, making contact with experimental studies on quasi\--\td{} colloidal systems.

Treating these diverse physical systems as complex networks allows a generalised network theory to be developed.
Traditional ring measures will be related to well\--defined metrics from network theory, namely the node degree distribution and the assortativity.
This will enable seemingly disparate physical systems to be compared and ties them into the wider field of network science.

Further applications of this work to real systems will be illustrated with analyses on a range of experimental (or experimentally motivated) data.
The appropriateness of weighted and unweighted Voronoi constructions will be evaluated for colloidal monolayers.
The ring structure of procrystalline lattices will be demonstrated to be related to the underlying parent lattice and atomic coordination environments.
Finally, a tool from topological data analysis, termed persistent homology, will be investigated and shown to be a potential complement to more traditional measures of structural disorder.

%\textbf{Keywords}: Network theory, Monte Carlo methods, silica bilayers, colloidal monolayers, procrystalline lattices, Voronoi diagrams, persistent homology
