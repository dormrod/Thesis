\chapter[Extension to Generic Physical Networks]{Extension to Generic Physical \\ Networks}
\label{ch:generalnetworks}

\begin{chapterabstract}
The properties of a wide range of physical \td{} networks are investigated by formulating a generalised network theory.
The methods developed are shown to be applicable to a wide range of systems generated from both computation and experiment; incorporating atomistic materials, foams, fullerenes, colloidal monolayers and geopolitical regions.
The ring structure in physical networks is described in terms of robust measures from network science: the node degree distribution and the assortativity.
These quantities are linked to previous empirical measures such as \lm's law and the Aboav\--Weaire law.
The effect on these network properties is explored by systematically changing the coordination environments, topologies and underlying potential model of the physical system.
\end{chapterabstract}

\section{Two\--Dimensional Networks in Nature}

So far this thesis has focussed on 3\--coordinate atomic networks such as silica and amorphous graphene.
These atomic systems can however be considered a subset of a much larger class of \td{} networks which occur throughout the natural world.
Such networks emerge across all disciplines and span many orders of magnitude in size.
In physical sciences random tessellations are not restricted to atomic materials, but are observed in foams, crack\--patterns  (in dessicated films, ceramics \etc) as well as in colloidal films through the Voronoi construction \-- to name a few \cite{Durand2011,Tong2017,Noever1992,Ma2019,Earnshaw1994,Allain1995,Moncho-Jorda2000}.
Similar mosaics can also be seen in the biological world in the form of epithelial cells and polymer networks such as collagen \cite{Honda1978,Carter2017,Kim2016,Broedersz2014}, as well as in geology in the guise of rock formations and geography in context of geopolitical borders \cite{Weaire1984,Goehring2014,LeCaer1993}.
Whilst this last example may seem to fall into the category of seemingly more esoteric offerings in the literature (including for instance crocodile scales and oil paintings \cite{Milinkovitch2019,Flores2017}), it provides an interesting insight into the formation of tessellations through random point processes.
Although man\--made maps are nominally carefully constructed, the influence of random geographical features serve to generate tessellations which are entirely consistent with others found in the natural world.

This is to say that the study of inorganic chemical networks fits into a wider remit of understanding the behaviours of generic physical networks.
Similarly the techniques and theory used to model and characterise atomic networks can be readily deployed to understand a wide range of other complex physical systems.
Therefore the focus of this chapter is on extending theory and computational methods to study general \td{} networks which are physically motivated (\ie{} have an underlying physical potential model). 
To demonstrate the effectiveness and potential of this approach, results will be compared to those from a wide variety of experimental systems.

\section{Generalised Network Theory}

A consequence of the universality of \td{} networks is that both the language and the metrics used to describe then varies considerably between fields, as demonstrated in table \ref{tab:genterms}.
From a nano\--materials perspective there are rings formed from a set of bonded atoms, in crystals there are grains separated by boundaries and in biological tissues cells which divide.
Further complication may arise from the concept of graph duality, where ring structure emerges only after transforming the physical coordinates.
In the context of colloidal monolayers for instance, rings are generated using the Voronoi construction; where the vertices have no real manifestation and the particle positions are the simplices in the dual Delaunay triangulation.
In addition as seen in previous chapters, there remains a prevalence of empirical laws to describe their structure.

\begin{table}[bt]
	\centering
     \caption{Terminology to describe ring structure in literature reflects the diversity of the underlying physical systems.}
     \label{tab:genterms}
     \begin{tabular}{cc}
     \toprule
     Term & Synonyms and Examples \\
     \midrule
     Ring & Face, polygon, cell, grain, pore, Voronoi cell\\
     Network & Graph, tiling, packing, tessellation, partition, \\
     & arrangement, decomposition, net, mosaic\\
     Link & Edge, bond, boundary, interface \\
     Node & Vertex, point, atom \\
     \bottomrule
     \end{tabular}
\end{table}

Network science offers an opportunity to unite the study of these disparate physical systems through a generalised theory.
Much of the groundwork for this has been laid in chapter \ref{ch:networktheory}, but there are some important additions, namely the introduction of the assortativity to describe ring\--ring correlations.
Firstly, some of the key aspects which were introduced in chapter \ref{ch:networktheory} will be briefly recapped and extensions highlighted.

Chapters \davidnote{link} of this thesis focussed on planar atomic systems which had a fixed coordination of three.
The main difference in this chapter is that the scope has increased to include networks with variable coordination and topologies.
For generic physical networks the equivalent of the atomic coordination number, $c$, is not necessarily precisely defined by an atomic species. 
The consequence of this is that the mean ring size as dictated by Euler's law is no longer always six, but rather determined by equations \eqref{eq:2dplanarcases},\eqref{eq:2dsphericalcases}; so that for example a network of $c=4$ will have a mean ring size of $\ki=4$.
That being said, the majority of naturally occurring networks still have $c=3$, as higher order sites are unstable with respect to small perturbations, with for example a 4\--coordinate site readily splitting into two 3\--coordinate sites \cite{Caer1993}.
In addition, the node degree distribution of the ring network, $p_k$, (equivalent to the ring statistics) is still an important measure and is expected to follow \lm's maximum entropy distribution \eqref{eq:mepk}.

XXX START TO TALK ABOUT AW XXX

\subsection{Deficiencies in the \aw{} Law}

\subsection{Assortativity as a Measure of Ring Correlations}







